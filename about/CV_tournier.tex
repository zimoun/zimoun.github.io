\documentclass[a4paper,10pt,twoside]{article}
\usepackage[utf8]{inputenc}
\usepackage[T1]{fontenc}
\usepackage[english,french]{babel}
\frenchspacing
\usepackage{aeguill}
%\usepackage{textcomp}
\usepackage{xcolor}
%\usepackage{latexsym}
%\usepackage{amsfonts}
%\usepackage{amssymb}
\usepackage{verbatim}
\usepackage{geometry}
\geometry{hmargin=50pt, vmargin=30pt}  % 60 et 70
%\usepackage{url}
\usepackage{hyperref}

\usepackage{multirow}

\definecolor{Orange}{rgb}{0.85,0.57,0.2}
\newcommand{\fin}{\\[1ex]}
\newcommand{\espace}{\vskip 1cm}

\hyphenation{ENSEEIHT Software}

% ----------------------------------------------
% Partie récupérée de Bertrand

% ----------------------------
% Titre de parties
\newcommand{\categorie}[1]{\vspace*{0.1cm}\noindent%
	{\center  \colorbox[gray]{0.9}{\makebox[\textwidth][c]{\Large
              \sc {#1}}}\par}\vspace*{.5cm}}
% ----------------------------
% Largeur des deux colonnes :
% Soit colunnun et colonndeux, pour les en-tetes
\newlength{\colonneun}
\settowidth{\colonneun}{BP adresse ???}
\newlength{\colonnedeux}
\setlength{\colonnedeux}{\textwidth}
\addtolength{\colonnedeux}{- \colonneun}
\addtolength{\colonnedeux}{-2\tabcolsep}
% Soit annee et texte pour les paragraphes standard annee - etudes (autres categories)
\newlength{\annee}
\settowidth{\annee}{\textbf{2007 -- 2010}}
\newlength{\texte}
\setlength{\texte}{\textwidth} \addtolength{\texte}{-\annee}
	\addtolength{\texte}{-2\tabcolsep}
\newlength{\anneelarge}
\settowidth{\anneelarge}{\textbf{Depuis Sept. 2007}}
\newlength{\textelarge}
\setlength{\textelarge}{\textwidth} \addtolength{\textelarge}{-\anneelarge}
	\addtolength{\textelarge}{-2\tabcolsep}
% \texte est la largeur de la deuxième colonne. Elle est définie comme
% étant la largeur de la page moins celle de la première colonne.
% 2\tabcolsep est la largeur de l'espacement entre les colonnes.
% ----------------------------


% ----------------------------------------------

% Partie adaptée de F. Clautiaux
% Première version : novembre 2004
% Dernière modification 03 janvier 2005

\newcommand{\entrylabel}[1]{\mbox{#1}\hfil}

\newenvironment{CV}
  {\begin{list}{}
    {\renewcommand{\makelabel}{\entrylabel}
      \setlength{\labelwidth}{77pt}
      \setlength{\leftmargin}{84pt}
    }
  }
  {\end{list}
  }


%%%%%%%%%%%%%%
%%% PUBLIS %%%
%%%%%%%%%%%%%%

 %% INITPUBLIS : A LANCER AVANT DE METTRE LES LISTES DE PUBLIS
 \newcommand{\initpublis}{
   \newcounter{cptpub}
   \setcounter{cptpub}{0}
 }

\newcommand{\initconf}{
   \newcounter{cptconf}
   \setcounter{cptconf}{0}
 }

\newcommand{\initconfss}{
   \newcounter{cptconfss}
   \setcounter{cptconfss}{0}
 }

 %% LISTPUBLIS : permet de cacher le truc bien compliqué ci-dessous
\newenvironment{listpublis}
  { \begin{CV} \item
    \begin{list}{[\arabic{enumi}]}{\settowidth\labelwidth{[10]}\topsep.4\topsep\itemsep-.2\baselineskip\leftmargin\labelwidth
         \advance\leftmargin\labelsep \usecounter{enumi}} \def\newblock{\hskip .11em
         plus .33em minus .07em} \sloppy\clubpenalty4000\widowpenalty4000
       \sfcode`\.=1000\relax\setcounter{enumi}{\value{cptpub}}
  }
  {\end{list}
   \end{CV}
  }

\newenvironment{listconf}
  { \begin{CV} \item
    \begin{list}{[\arabic{enumi}]}{\settowidth\labelwidth{[10]}\topsep.4\topsep\itemsep-.2\baselineskip\leftmargin\labelwidth
         \advance\leftmargin\labelsep \usecounter{enumi}} \def\newblock{\hskip .11em
         plus .33em minus .07em} \sloppy\clubpenalty4000\widowpenalty4000
       \sfcode`\.=1000\relax\setcounter{enumi}{\value{cptconf}}
  }
  {\end{list}
   \end{CV}
  }

\newenvironment{listconfss}
  { \begin{CV} \item
    \begin{list}{[\arabic{enumi}]}{\settowidth\labelwidth{[10]}\topsep.4\topsep\itemsep-.2\baselineskip\leftmargin\labelwidth
         \advance\leftmargin\labelsep \usecounter{enumi}} \def\newblock{\hskip .11em
         plus .33em minus .07em} \sloppy\clubpenalty4000\widowpenalty4000
       \sfcode`\.=1000\relax\setcounter{enumi}{\value{cptconfss}}
  }
  {\end{list}
   \end{CV}
  }

% \newcommand{\publication}[4]{
%       #1,
%       \emph{#2},
%       #3, #4 \\
%       \stepcounter{cptpub}
% }


% UNE PUBLICATION
% CLE, auteurs, titre, nom de la revue / conf, date
\newcommand{\publi}[5]{
  \bibitem{#1}
      #2,
      \emph{#3},
      #4, #5 \\
      \stepcounter{cptpub}
}

\newcommand{\conf}[5]{
  \bibitem{#1}
      #2,
      \emph{#3},
      #4, #5 \\
      \stepcounter{cptconf}
}

\newcommand{\confss}[5]{
  \bibitem{#1}
      \emph{#2},
      #3, #4 \\
      \stepcounter{cptconfss}
}

\initpublis
\initconf
\initconfss

% exemple d'utilisation
% ---------------------
%
% \begin{listpublis}
% \publi{PUB92}{Jacques Dupont}{Titre de ma publi}{Référence du bouquin, article, conf}{date}
% \publi{PUB92B}{Jacques Dupont}{Titre de ma publi 2}{Référence du bouquin, article, conf}{date}
%\end{listpublis}
% ...
% \begin{listpublis}
% \publi{SYS32}{Jacques Dupont}{Titre de ma publi 3}{Référence du bouquin, article, conf}{date}
% \publi{ARJ45}{Jacques Dupont}{Titre de ma publi 4}{Référence du bouquin, article, conf}{date}
%\end{listpublis}

% Plus besoin de gérer la numérotation !

% ----------------------------------------------

\newcommand\moi{$\underline{\textrm{S. Tournier}}$}
\newcommand\moistar{$\underline{\textrm{S. Tournier}}^\star$}

\begin{document}
\pagestyle{empty}


\noindent
\begin{tabular}{@{} l @{\quad} r}
  \begin{minipage}[t]{0.55\linewidth}
    \textbf{{\Large Simon \textsc{Tournier}}\\[1ex] }
    Born the 23${}^{rd}$ June 1983 in Montpellier (France)\\
    French\\
  \end{minipage}
  &
  %\colorbox{Orange}{
  \begin{minipage}[t]{0.45\linewidth}
    \begin{flushright}
      Université Paris 7 Diderot\\
      BioData Center\\
      Institut Universitaire d'Hématologie\\
      Hôpital Saint-Louis\\
    \end{flushright}
  \end{minipage} %}
\end{tabular}

%% \medskip
%\vspace{0.1cm}

\noindent
\begin{tabular}{@{} l l}
  \textit{Tél. :} & +33 (0) 6 12 32 19 52  \\
  \textit{Email :} & \url{ simon.tournier@alumni.enseeiht.fr} \\
\end{tabular}

%\medskip
\vspace{-0.3cm}
\begin{center}
  %\line(1,0){250}
  %\\
  \vspace{0.1cm}
  \fbox{
    \begin{minipage}{0.7\textwidth}
      \begin{center}
        Modeling and Analysis in Computational Electromagnetism and Acoustic,\\
  Preconditionning techniques,
  Homogenization,
  Domain Decomposition Method,\\
  Scientific Computing
      \end{center}
    \end{minipage}
  }
  %\\
  %\line(1,0){250}
\end{center}

\vspace{-0.5cm}
\categorie{Academic Background and Experiences}

\noindent
\begin{tabular}{r @{\qquad} p{\textelarge}}

  \textbf{2016 -- \dots} &
  \begin{minipage}[t]{\linewidth}
    \textbf{Research Engineer} position at the Université Paris 7 Diderot\\
    \normalsize
    in charge of numerical Core Facilities in biological wet laboratory:\\
    $\bullet$ support about bioinformatics tools:
    predictive modeling,
    clustering analysis of flow cytometry data,
    aligment of Next Generation Sequencing (NGS) data and variant calling
    ;\\
    $\bullet$ system administrator of 9 nodes cluster and of desktop computers,
    management of large data sets from biological experiments.
    %% \begin{itemize}
    %% \item support about bioinformatics tools
    %% \item system administrator of 9 nodes cluster
    %% \end{itemize}
    \\[-1ex]
  \end{minipage} \\


  \textbf{2014 -- 2016} &
  \begin{minipage}[t]{\linewidth}
    \textbf{Post-doctoral} position at the PUC (Chile)
    \small{[FONDECYT grant: 3150446]} \\
    \normalsize
    under the supervision of \textsf{Prof. Carlos Jerez-Hanckes},\\
    \emph{Efficient and Robust HPC Solver
      for Multiple Traces Formulations}\\
       \emph{for Engineering Applications}. \\[-1ex]
  \end{minipage} \\


  \textbf{2012 -- 2013} &
  \begin{minipage}[t]{\linewidth}
    \textbf{Post-doctoral} position at the Université de Liège (Belgium),
    in the ACE team,\\
    under the supervision of \textsf{Prof. Christophe Geuzaine},\\
    \emph{Study of some preconditioning techniques
      for Finite Elements Methods}\\
    \emph{and Decomposition of Domain Method}.  \\[-1ex]
  \end{minipage} \\

  \textbf{2007 -- 2012} &
  \begin{minipage}[t]{\linewidth}
    \textbf{PhD}
    from Institut Supérieur de l'Aéronautique et de l'Espace (ISAE),
    Toulouse,\\
    under the supervision of
    \textsf{Pierre Borderies} (ONERA, Toulouse)\\
    and \textsf{Jean-René Poirier} {(LAPLACE, Toulouse)}\\ %,
    Defended the 22${}^{nd}$ March 2012 at SupAéro (ISAE),
    with the jury composed by :
    Abderrahmane Bendali, Pierre Borderies, Christophe Bourlier,
    Christophe Geuzaine, Luc Giraud, Jean-René Poirier,
    Jean-Yves Suratteau.\\
    \textbf{Title :}
    \emph{Contribution of the modeling
      of the electromagnetic scattering}\\
      \emph{by rough surfaces
      from rigorous methods}.\\
  \end{minipage} \\[1ex]

  %% 2007 -- 2011 &
  %% \begin{minipage}[t]{\linewidth}
  %%   \textbf{Teaching}
  %%   in the Department of Electronics and Signal Processing,
  %%   ENSEEIHT, Toulouse :
  %%   \begin{itemize}
  %%   \item Introduction to the Analysis of Partial Differential
  %%     Equations \hfill \emph{(master level)},
  %%   \item Fourier Analysis  \hfill  \emph{(undergraduate level)},
  %%   \item Numerical Analysis  \hfill  \emph{(undegraduate level)},
  %%   \item Algorithm and Programming in C  \hfill \emph{(undergraduate. level)}.
  %%   \end{itemize}
  %%   I also supervised several students in projects  \hfill \emph{(Bachelor level)}:
  %%   \begin{itemize}
  %%   \item Study of an equivalent impedance of a rough surface,
  %%   \item Comparison between plane waves and Gaussian beams in a MoM code,
  %%   \item Numerical effects of the finitude of surfaces in the spectrum of
  %%   integral operators.\\
  %%   \end{itemize}
  %% \end{minipage}\\ [1ex]

  2007 &
  \begin{minipage}[t]{\linewidth}
    7 months in EADS Innovation Works
    (Centre Commun de Recherches)\\
    Engineer intern under the supervision of \textsf{Andrew Thain}.\\
  \end{minipage}\\

  2006--2007 &
  \begin{minipage}[t]{1.0\linewidth}
    \textbf{Master of Science} (\emph{magna cum laude})
    in ``ElectroMagnetism and OptoElectronics'',\\
    Institut National Polytechnique, Toulouse. \\
    Thesis under the surpervision of \textsf{Andrew Thain} (EADS Innovation Works),\\
    \emph{Numerical Simulations of antennas on large planes}. \\[-1ex]
  \end{minipage} \\[1ex]

  2005 &
    \begin{minipage}[t]{\linewidth}
      9 weeks at Dublin City University, Radio and Optical Comm. Lab.,\\
      under the supervision of Frédéric Surre and Prof. Pascal Landais,\\
      \emph{Numerical Investigations of Losses in THz waveguides}.\\
    \end{minipage}\\

  \textbf{2004 -- 2007} &
  \begin{minipage}[t]{\linewidth}
    \textbf{Engineer degree} in Electronics and Signal Processing,\\
    ENSEEIHT, Toulouse. \\
  \end{minipage} \\

  2001--2004 &
  \begin{minipage}[t]{\linewidth}
    Preparatory Class for entrance in engineering school, Montpellier. \\
    \small{\emph{Personal Project: Modeling of 1D snow avalanche and
        numerical simulaion by finite difference}}.
  \end{minipage} \\

\end{tabular}


\medskip

\categorie{Publications}
\vspace{-0.65cm}
\subsubsection*{Articles published under peer-review}

\begin{itemize}
\item[\textbullet]
\emph{Integral Equations Physically~based Preconditioner for Two~Dimensional Electromagnetic Scattering by Rough Surfaces}, \\
 \moi , P. Borderies, J.-R. Poirier \\
\textsf{IEEE Antennas and Propagation},
Vol. 59, No. 10, pp. 3764-3774, oct. 2011. \\[-1ex]
\item[\textbullet]
\emph{Modélisation de la diffusion électromagnétique par des surfaces
  rugueuses à partir de méthodes rigoureuses},\\
\moi , P. Borderies, J.-R. Poirier \\
\textsf{R}evue d'\textsf{E}lectricité et \textsf{E}lectronique,
No. juin 2012.\\
(request by the journal for section ``Jeunes Chercheurs'')
\item[\textbullet]
\emph{Local Multiple Traces Formulation for High-Frequency Scattering Problems},\\
 C. Jerez-Hanckes , J. Pinto, \moi \\
 \textsf{Journal of Computational and Applied Mathematics},
 Vol. 289, pp. 306-321, dec. 2015. %[-1ex]
\item[\textbullet]
 \emph{Local Multiple Traces Formulation for High-Frequency Scattering
   Problems by Spectral Elements},\\
 C. Jerez-Hanckes , J. Pinto, \moi \\
\textsf{Scientific Computing in Electrical Engineering: SCEE 2014,
  Wuppertal, Germany},
series Mathematics and Industry, Springer,
 pp. 73-82, 2016
\item[\textbullet]
  \emph{\texttt{GetDDM}: an Open Framework for Testing Optimized Schwarz
    Methods for Time-Harmonic Wave Problems},\\
  B. Thierry, A. Vion, \moi, M. El Bouajaji,
  D. Colignon, N. Marsic, X. Antoine, C. Geuzaine\\
  \textsf{Computer Physics Communications},
  Vol. 203, pp. 309-330, 2016
  \begin{flushright}
    (see \url{http://onelab.info/wiki/GetDDM} )
  \end{flushright}
\end{itemize}

%% \subsubsection*{Article submitted}
%% \begin{itemize}
%% \end{itemize}

\vspace{-0.65cm}

\subsubsection*{Article submitted}
\begin{itemize}
\item[\textbullet]
  \emph{Technique of Homogenization to Improve the Scattering by one-dimensional
    Rough Surface}\\
  \moi, J.-R. Poirier, P. Borderies\\
  \textsf{IEEE Antennas and Propagation}
  \\
\end{itemize}

\vspace{-0.65cm}

\subsubsection*{Article in preparation}
\begin{itemize}
\item[\textbullet]
  \emph{Multi-Scattering with Transmission Conditions: efficient
    preconditionned multi-trace formulation},\\
  with \textsf{C. Jerez-Hanckes}.
\end{itemize}

\vspace{-0.45cm}

\subsubsection*{International Conferences (with committee selection)}
\noindent
\begin{itemize}
\item[\textbullet]
  \textbf{SIAM} \textbf{2016} Annual Meeting, Boston\\
  \emph{Multiple Traces Formulations: Novel Extensions and Challenges} ;
  C. Jerez-Hanckes, \moi

\item[\textbullet]
  \textbf{FACM} \textbf{2016}, Newark\\
  \emph{Multiple Traces Formulation: Preconditioning Strategies} ;
  C. Jerez-Hanckes, \moi

\item[\textbullet]
  \textbf{WAVES} \textbf{2015}, Karlsruhe,\\
  \emph{Preconditioning Techniques
    for Local Multiple Traces Formulation for Scattering Problems} ;
  \moistar,  J. Pinto, C. Jerez-Hanckes

\item[\textbullet]
  \textbf{WAVES} \textbf{2015}, Karlsruhe,\\
  \emph{Local Multiple Traces Modelling for High-Frequency Scattering} ;
  C. Jerez-Hanckes, J. Pinto, \moi

\item[\textbullet]
  \textbf{PANACM} \textbf{2015}, Buenos Aires,\\
  \emph{Multiple Traces Formulation for High-Frequency Scattering} ;
  C. Jerez-Hanckes, J. Pinto, \moi

\item[\textbullet]
  \textbf{IEEE ACAMA} \textbf{2014}, Antibes Juan-les-Pins,\\
  \emph{An Open Source Domain Decomposition Solver for
    Time-Harmonic Electromagnetic Wave Problems} ;
  C. Geuzaine, B. Thierry, N. Marsic, D. Colignon, A. Vion, \moi,
  Y. Boubendir, M. El Bouajaji, X. Antoine

\item[\textbullet]
  \textbf{SCEE} \textbf{2014}, Wuppertal,\\
  \emph{Local Multiple Traces Formulation for High-Frequency Scattering
    Problems} ;
  C. Jerez-Hanckes , J. Pinto, \moi

\item[\textbullet]
  \textbf{EuroEM} \textbf{2012}, Toulouse,\\
  \emph{Homogenization  Techniques for Improving Electromagnetic
    Scattering Computation by Dielectric Surfaces} ;
  \moistar, P. Borderies, J.-R. Poirier

\item[\textbullet]
  \textbf{AMPERE}
  %\footnote{13th International Conference
  %on Microwave and RF Heating}
  \textbf{2011}, Toulouse
  -- \textsf{Best Poster Award}\\
  \emph{Analysis of QR-compression
    Techniques for Improving Electromagnetic Scattering Computation by
    Periodic Rough Surfaces} ;
  \moistar, J. Girardin, J.-R. Poirier, P. Borderies

\item[\textbullet]
  \textbf{PIERS}
%\footnote{Progress In Electromagnetics Research
%   Symposium}
  \textbf{2010}, Cambridge, \\
  \emph{Analysis of Homogenization Techniques for Improving
    Electromagnetic Scattering Computation by Rough Surfaces} ;
  \moistar, P. Borderies, J.-R. Poirier

\item[\textbullet]
  \textbf{WAVES}
  %\footnote{9th International Conference on Mathematical
  %  and Numerical Aspects of Waves Propagation}
  \textbf{2009}, Pau,\\
  \emph{A Physically-based Preconditioner for 2D Electromagnetic Rough
    Surfaces Scattering Problems} ;
  \moistar, P. Borderies, J.-R. Poirier

\item[\textbullet]
  \textbf{WAVES 2009}, Pau,\\
  \emph{High order asymptotic expansion for the scattering of fast
    oscillating periodic surfaces} ;
  J.-R. Poirier, A. Bendali, P. Borderies,  \moi

\item[\textbullet]
  \textbf{PIERS 2009}, Beijing,\\
  \emph{Analysis of Performances of a Floquet Mode Preconditioner for
    Electromagnetic Scattering Computation by Rough Surfaces} ;
  \moi, J.-R. Poirier, P. Borderies

\item[\textbullet]
  \textbf{PIERS 2008}, Hangzhou,\\
  \emph{Use of Numerical Methods for Assessing Validity Domains of the
    approximations Involved in Electromagnetic Interaction Modeling
    with vegetation} ;
  P. Borderies, J.-R. Poirier, \moi, C. Lauprette, L. Villard, P. Dubois~Fernandez, N. Floury
\end{itemize}

\vspace{0.1cm}

\noindent
\textbf{Reviewer} for
IEEE Antennas and Propagation,
IEEE Geoscience and Remote Sensing


\categorie{Computer Skills}
\noindent

\begin{tabular}{crl}
  \multirow{5}{*}{\textbf{Scientific Programming}}
  & \textbf{current daily use:} &
  Python, \quad R, \quad \texttt{bash}
  \\
  & \textbf{librairies:} &
  Numpy/Scipy, \hspace{\stretch{1}} BLAS/Lapack, \hspace{\stretch{1}} PETSc (MPI)
  \\
  & \textbf{previously used:} &
  \texttt{C}, \quad \texttt{Fortran}, \quad \texttt{C++}, \quad \textsc{Matlab}/Scilab
  \\
  & \textbf{basic knowledge:} &
  Julia, \quad Haskell, \quad OCaml, \quad Lisp
  \\
  & \textbf{advanced user:} &
  Gmsh, %\footnote{\url{http://gmsh.info}},
  \quad
  GetDP, %\footnote{\url{http://getdp.info}},
  \quad
  {Bem++}%\footnote{\url{http://www.bempp.org}},

  \\
  &
  \\

  \multirow{5}{*}{\textbf{Tools}}
  & \textbf{visualizing:} &
  Matplotlib, \quad {ggplot}
  \\
  & \textbf{editing:} &
  \LaTeX/\textsc{Bib}\TeX, \quad Markdown, \quad Org, \quad Emacs
  \\
  & \textbf{version control:} &
  \texttt{git}, \quad mercurial, \quad subversion
  \\
  & \textbf{debug:} &
  \texttt{gdb}, \quad \texttt{pdb}, \quad {Valgrind}, \quad \texttt{gprof}
  \\
  & \textbf{build automation:} &
  Makefile, \quad CMake, \quad Continuous Integration (TravisCI)
  \\

\end{tabular}




\categorie{Others}
\noindent

\begin{tabular}{ll}
  \begin{minipage}{0.3\linewidth}
    \textbf{voluntary of GENEPI}\\
    (from 2004 to 2009)\\
    \url{http://www.genepi.fr}
  \end{minipage}
  &
  \begin{minipage}{0.7\linewidth}
  Intervention in prison\\
  \emph{(teaching, participation to an internal newspaper, sports)},\\
  Organization of events to talk about problems of prison\\
  \emph{(intervention in high school, conferences, radio emission)}
  \end{minipage}\\

  \\

  \begin{minipage}{0.3\linewidth}
  participation to Colombbus\\
  \url{http://www.colombbus.org}
  \end{minipage}
  &
  \begin{minipage}{0.7\linewidth}
  Promotion of computer sciences
  in junior secondary school using Free Software
  \end{minipage}\\

  \\

  {Miscellaneous}
  & Mountain (hiking, climbing)
  \\& \small{user of GNU/Linux since 1999.}
\end{tabular}

\categorie{References}
\begin{tabular}{lcl}
  \begin{minipage}{0.5\linewidth}
    \quad \textbf{Jean-René \textsc{Poirier}}\\
    LAPLACE -- INPT-ENSEEIHT\\
    2 rue Charles Camichel, BP 7122\\
    FR-31071 Toulouse, Cedex 7, France\\
    \url{poirier@laplace.univ-tlse.fr}\\
    +33 5 343 223 81
  \end{minipage}
& &
 \begin{minipage}{0.5\linewidth}
    \quad\textbf{Pierre \textsc{Borderies}}\\
    ONERA -- DEMR\\
    2 avenue Edouard Belin, BP 74025\\
    FR-31055 Toulouse, Cedex 4, France\\
    \url{pierre.borderies@onera.fr}\\
    +33 5 622 527 18
  \end{minipage}
\\
& &
\\
  \begin{minipage}{0.5\linewidth}
    \quad\textbf{Christophe \textsc{Geuzaine}}\\
    University of Liège --  Montefiore Institute\\
    Sart-Tilman, B28, P32\\
    B-4000 Liège, Belgium\\
    \url{cgeuzaine@ulg.ac.be}\\
    +32 4 366 37 30
  \end{minipage}
  & &
  \begin{minipage}{0.5\linewidth}
    \quad\textbf{Carlos \textsc{Jerez-Hanckes}}\\
    Pontificia Universidad Cat\'olica de Chile\\
    Av. Vicuna Mackenna 4860, Macul\\
    Santiago de Chile, (Postal Code) 7820436, Chile\\
    \url{cjerez@ing.puc.cl}\\
    +56 22 552 2563
  \end{minipage}

 %% \begin{minipage}{0.5\linewidth}
 %%    \quad\textbf{Xavier \textsc{Antoine}}\\
 %%    Université  de Lorraine\\
 %%    Bureau 301, B.P. 239\\
 %%    FR-54506 Vandoeuvre-lès-Nancy Cedex, France\\
 %%    \url{xavier.antoine@univ-lorraine.fr}\\
 %%    +33 3 836 845 61
 %%  \end{minipage}
\end{tabular}

\end{document}
